\documentclass[french, utf8]{article}
\usepackage[utf8]{inputenc}
\usepackage[T1]{fontenc}
\usepackage[french]{babel}
\usepackage[parfill]{parskip}
\usepackage{amsmath}
\usepackage{amssymb}
\usepackage{amsfonts}
\usepackage{graphicx}
\usepackage{subfigure}
\usepackage[font={small}]{caption}
\usepackage{float}
\usepackage{listingsutf8}
\usepackage{fullpage}
\usepackage[nochapter]{vhistory}
\usepackage{hyperref}

% -----------------------------------------------------
% -----------------------------------------------------
% -----------------------------------------------------

\hypersetup{
%couleurs des liens cliquable changée pour une meilleur lisibilité
    colorlinks=true,
    linkcolor=blue,
    filecolor=magenta,
    urlcolor=cyan,
    pdfpagemode=FullScreen,
    }

\title{INFO-F209 - Projet d'informatique 2 }
\author{Anton Romanova, Mohammad Secundar, Esteban Aguililla Klein, Vlad Moruntale, Mathieu Van Den Bremt, Nabil Abdellaoui, Ayman Boulaich}
\date{Novembre 2021}

\begin{document}
\maketitle
\tableofcontents
\newpage


% -----------------------------------------------------
% -----------------------------------------------------
% -----------------------------------------------------
\section{Introduction}
% -----------------------------------------------------
\subsection{But}
Faire un portage du jeux de plateau classique multijoueur Quoridor\footnote{les règles détailles peuvent être consultée dans la section annexe} par le biais d'une interface client-serveur et l'ajout de fonctionnalité sociale modernes.
\\ \\
Pour que la partie se termine, dans le mode classique, il suffit d'atteindre le côté opposé du plateau tout en empêchant l'adversaire de faire de même à l'aide de murs plaçables.    %TODO: dans le mode X, il faut Y
\\ \\
Concernant les fonctionnalités sociales, il est possible de gérer une liste d'amis, discuter avec ces derniers, créer une partie privées et consulter un classement des joueurs.
\\ \\
De par sa nature simple, le jeux se veut tout public malgré ses fonctionnalités sociales non modérées.  %ressemblance avec réseau social donc >13 ans min ???

% -----------------------------------------------------
\subsection{Glossaire}

% -----------------------------------------------------
% AJOUTEZ TOUS VOS CHANGEMENTS ICI
\subsection{Historique du document}

\begin{versionhistory}
\vhEntry{0.1.3}{19.11.21}{Bourgeois Noé }{Lancement, enregistrement, créer ou rejoindre une partie, rencontre inopinée de alex ellioti}
\vhEntry{0.1.6}{20.11.21}{Mathieu Van Den Bremt \& Nabil Abdellaoui}{Modification des UseCase et amélioration divers pour User requirements}
\vhEntry{0.1.5}{19.11.21}{Aguililla Klein Esteban \& Moruntale Vlad}{Ajout du but et d'un annexe}
\vhEntry{0.1.4}{19.11.21}{Mathieu Van Den Bremt }{Tableau Use Case}
\vhEntry{0.1.3}{19.11.21}{Boulaich Ayman }{Sous section des besoins fonctionnels du système et début }
  \vhEntry{0.1.2}{18.11.21}{Anton Romanova}{Ajouts de la section "Annexes" et "Design"}
   \vhEntry{0.1.1}{17.11.21}{Mathieu Van Den Bremt \& Nabil Abdellaoui}{Début Besoin d'utilisateur + diagrammes Use Case}
  \vhEntry{0.1}{16.11.21}{Anton Romanova}{Structure générale}
\end{versionhistory}

% -----------------------------------------------------
% -----------------------------------------------------
% -----------------------------------------------------
\newpage
\section{Besoin d'utilisateur}
% -----------------------------------------------------

\subsection{Besoins fonctionnelles}
\subsubsection{Inscription et connexion}
L'utilisateur doit être capable de présenter un nom de compte ainsi qu'un mot de passe au démarrage du jeu pour pouvoir se connecter et accéder à son menu principal. Si il n'a pas de compte ou si il désire en recréer un, le jeu peut proposer à l'utilisateur d'en créer un nouveau. Pour la création d'un compte, aucun mail ne sera nécessaire à introduire, l'utilisateur devra juste présenter un nouveau pseudonyme accompagné d'un mot de passe pour terminer le processus de création. \newline


\begin{figure}[ht]
     \centering
    %\includegraphics[width=70mm,scale=0.1]{Image/SystemUC.PNG}
    \includegraphics[width=150mm,scale=0.1]{Image/ConnectionUC.png}

\end{figure}
\begin{center}
\begin{tabular}{|m{3cm}|m{3cm}|m{3cm}|m{3cm}|m{3cm}|}
\hline  Use Case & Pré condition      &  Post condition  & Cas gérnéral & Cas exceptionnels\\
\hline Register& l'utilisateur n'a pas  encore un compte & On enregistre le nouveau compte & Le programme demande un nom et un mot de passe & Si le nom est déjà utilisé alors on envoie une erreur  \\
\hline Connect  & L'utilisateur a déjà un compte & L'utilisateur rentre dans le programme & Le programme vérifie les informations donnée par le client ensuite, il lui permet de continuer sur le programme & Si les informations sont incorrects, alors on envoie une erreur \\
\hline
\end{tabular}\\
\end{center}
\subsubsection{Écran Principal}
Après connexion, l'utilisateur aura accès à différentes fonctionnalités. Tout d'abord, une multitude d'interactions utilisant un réseau lui sera disponible. Il pourra gérer une liste d'amis et/ou discuter avec eux en s'échangeant des messages, il pourra aussi accéder à un  classement des meilleurs joueurs. Et c'est bien à partir du menu que l'utilisateur pourra créer et lancer une partie. \newline


Avant de pouvoir jouer, l'utilisateur pourra configurer sa partie en modifiant plusieurs paramètres. Il devra indiquer le nombre de participants qui seront de la partie, ils pourront être 2 ou 4 à joueurs. Par la suite, il devra indiquer quels joueurs, parmi sa liste d'amis, rejoindront la partie, les joueurs invités devront confirmer leur participation. Une fois tout ceci fait, les joueurs rejoindront automatiquement un unique salon et seront tous redirigés vers la partie nouvellement lancée. \newline

Si le cours d'une partie a été précédemment sauvegardé, l'utilisateur pourra reprendre cette même partie.

Si il le désire, l'utilisateur pourra demander de l'aide, l'application affichera le fonctionnement du jeu et les différentes fonctionalités de cette dernière.

- peut ensuite avoir  la possibilité de discuter avec des gens dans une liste d’amis qu’il peut gérer
\newline

\begin{figure}[ht]
     \centering
    %\includegraphics[width=70mm,scale=0.1]{Image/SystemUC.PNG}
    \includegraphics[width=150mm,scale=0.1]{Image/MainScreenUC.png}

\end{figure}
\begin{center}
\begin{tabular}{|m{3cm}|m{3cm}|m{3cm}|m{3cm}|m{3cm}|}
\hline  Use Case & Pré condition      &  Post condition  & Cas gérnéral & Cas exceptionnels\\
\hline Check Ranking & Un classement existe  & L'utilisateur voit le classement & Le programme montre le classement des joueurs & Néant \\
\hline Manage friend list & Néant & La list est modifiée si l'utilisateur le veut & Le programme ouvre la liste d'amis & Néant \\
\hline Send Message & Utilisateur cible existe & Le message est envoyé & Le programme envoie un message écrit par l'utilisateur à une personne de la liste d'amis de celui-ci & Néant \\
\hline Add Friend & L'ami qui doit être ajouter à la liste existe & La liste est modifiée & Le programme sauvegarde le contact du nouvel ami dans la liste & Si le nouvel ami n'existe pas  alors on envoie une erreur à l'utilisateur \\
\hline Delete Friend & L'ami qui doit être supprimé de la liste ést dans la liste & La liste est modifiée & Le programme supprime le contact de l'ami dans la liste & Si l'ami n'est pas dans la liste alors on envoie une erreur à l'utilisateur \\
\hline Create Game & Néant & Néant & Le programme lance la configuration d'une partie puis lance celle ci après avoir lancer l'invitation d'un joueur & Néant \\
\hline Configure Game & Une partie est sur le point d'être créée  & Les paramètres de la partie change & Le programme permet à l'utilisateur de choisir les paramètres de sa partie & Néant \\
\hline Invite Friend to Game & L'ami invité doit faire partie de la liste d'ami & L'ami rejoinds la partie & Le programme permet à l'utilisateur d'inviter un ami à jouer avec lui & Si l'ami refuse l'invitation, le programme demande à l'utilisateur d'inviter un autre ami \\
\hline
\end{tabular}\\
\end{center}
\newpage
\begin{center}
\begin{tabular}{|m{3cm}|m{3cm}|m{3cm}|m{3cm}|m{3cm}|}
\hline Join Game by invitation & Un autre utilisateur a invité l'utilisateur & L'utilisateur rejoinds la partie & Le programme invite l'utilisateur à rejoindre une partie & Si l'utilisateur refuse alors le programme envoie un message à l'utilisateur ayant envoyé l'invitation \\
\hline Joined Saved Game & Une partie sauvegardée existe et l'utilisateur qui a joué précédement accepte de jouer & La partie sauvegardée est lancée & Le programme reprends la partie sauvegardée  & Si l'autre utilisateur refuse ou si la partie sauvegardée n'existe pas ou ne peut être lancée, le programme renvoie une erreur \\
\hline Play Game & Néant & En fin de partie les résultat sont mis à jour dans le classement & Le programme démarre la partie & Si la partie est quittée en cours de jeu, alors le classement n'est pas mis à jour \\
\hline Get Help & Néant & Néant & Le programme affiche l'aide pour le programme et les règles du jeu & Néant \\
\hline
\end{tabular}\\
\end{center}
\subsubsection{Durant une partie}
Quand c'est son tour, le joueur doit pouvoir effectuer une action, déplacer un pion ou mettre un mur. Cette action est représenté comme un message envoyé à l'application qui agira sur lui même en conséquence.
En plein duel, l'un des joueurs peut proposer au reste des participants de mettre le jeu en pause et de sauvegarder la partie en cours pour pouvoir la continuer plus tard. Les joueurs peuvent aussi déclarer forfait et donc se retirer.
Si l'un des joueurs se déconnecte sans proposer de sauvegarder la partie, il est considéré comme disqualifié, ses murs posés seront toujours présents mais ses pions se retrouveront retirés du plateau. \newline

\begin{figure}[ht]
     \centering
    %\includegraphics[width=70mm,scale=0.1]{Image/GameUC.PNG}
    \includegraphics[width=110mm,scale=0.1]{Image/GameUC.png}

\end{figure}
%ligne vide tableau : \hline .  & . & . & . &. \\
\begin{center}
\begin{tabular}{|m{3cm}|m{3cm}|m{3cm}|m{3cm}|m{3cm}|}
\hline  Use Case & Pré condition      &  Post condition  & Cas gérnéral & Cas exceptionnels\\
\hline Play Game& L'utilisateur à lancer et configurer une partie et il y a un nombre requis de Joueur & A la fin de la partie on retourne sur l'écran principal & Le programme lance le jeu Quoridor et demande tour par tour, quel actions les joueur veulent faire & Si un joueur quitte en cours de partie un message est envoyé à l'adversaire  \\
\hline Move pawn  & Le pion peut être déplacer comme le veut l'utilisateur & Le plateau est modifié avec la nouvelle position du pion & Le programme déplace le pion & Néant \\
\hline Place Wall  & L'emplacement désigné pour le mur est libre et ne bloque pas entièrement un pion & Le plateau est modifié avec le nouveau mur & Le programme place un mur dans la position choisie & Néant \\
\hline Pause Game  & Néant & Le jeu est mis en pause, sauvegardé et ensuite terminé & Le programme demande à l'adversaire si celui-ci accepte de mettre en pause le jeu & Néant \\
\hline Save  & Le jeu a été mis en pause sur l'accord des deux joueurs & La partie est sauvegardée & Le programme sauvegarde la partie tel qu'elle est & Si le programme n'a pas réussi à sauvegarder les données, alors celui-ci envoie un message d'erreur \\
\hline Forfeit  & Néant & La partie est terminée sur une victoire adverse & Le programme termine la partie & Néant \\
\hline
\end{tabular}\\
\end{center}
% -----------------------------------------------------
\newpage
\subsection{Exigences non-fonctionnelles}
\subsubsection{Compte}
Le programme doit vérifier si le compte présenté au démarrage de l'application existe. Si le pseudonyme inséré n'est pas valide ou si le mot de passe n'est pas le bon, l'application empêche tout accès au menu principal.  \newline
L'utilisateur peut néanmoins en créer un nouveau. Si, lors d'un enregistrement, un nouveau pseudonyme est entré et n'est pas reconnu par l'application depuis sa base de donnée, celui-ci pourra être utilisé pour créer un nouveau compte, l'application agit ainsi pour ne pas créer de doublons. \newline
\subsubsection{Interface}
La version simpliste de ce programme affiche un plateau de jeu en deux dimensions dans un terminal de commande et propose des modèles et des couleurs assez limités. \newline
Par après, une version beaucoup plus performante et présentable sera affiché grâce à une interface graphique.
\subsubsection{Latence}
Toutes les actions doivent passer d’un utilisateur à un autre avec une latence minimale pour avoir l’air instantanée. Ceci dit, c'est un paramètre qui dépend énormément de la qualité internet des joueurs participants.
\subsection{Légalité}
L'utilisateur doit pouvoir supprimé son compte selon le GDPR.


% -----------------------------------------------------
% -----------------------------------------------------
% -----------------------------------------------------
\section{Besoins du système}

% -----------------------------------------------------
\subsection{Besoins fonctionnels}
Use case diagram, pouvoir se connecter, etc...
Ke système doit, l'utilisateur peut,
Les besoins fonctionnels du système

Un utilisateur peut être connecté simultanément avec un même compte avec plusieurs machines.
\subsubsection{Lancement}
Le programme, à son lancement, demande à l’utilisateur d'entrer
\\un pseudo et
\\un mot de passe.

il peut ensuite
\\soit créer un nouveau compte,
\\soit se connecter avec un compte déjà enregistré dans une base de
donnée.

\subsubsection{Enregistrement}
\\ Si le pseudo  n’est pas déjà utilisé, le compte et le mot de passe sont enregistrés dans
la base de données.

\subsubsection{Connexion}
Lorsque l'utilisateur lance le programme il devra se connecter via un nom d'utilisateur et un mot de passe s'il  possède un compte . Si l'utilisateur ne possède pas de compte il aura la possibilité de s'enregistrer et ainsi de faire en sorte que son compte soit stocker dans la base de donnée.
%---------------------------------------------
\newline
\\Si l’utilisateur demande à se connecter,
        \\Si le pseudo se trouve dans la base de donnée,
            \\et le mot de passe correspond,
                \\le programme établit la connection entre l'utilisateur et le serveur
                puis affiche le menu principal.
        \\Si le pseudo ne se trouve pas dans la base de données,
            \\il avertit l'utilisateur que ce compte n'est pas enregistré.
        \\Si le pseudo s'y trouve, mais le mot de passe ne correspond
pas,
            il avertit l'utilisateur en conséquence.
\newline
%-------------------------------------------
use case
supp compte
Question: quelle moyen utilisé pour retrouver son compte avec mot de passe oublié ? question personnelle

\subsubsection{Gestion du compte}
L'utilisateur pourra modifier son compte lorsqu'il est connecté via ce dernier au programme.L'utilisateur aura plusieurs possibilités lors de la modification de son compte tels que changer sa photo de profil , son nom d'utilisateur, son mot de passe , ajouter une description .

\subsection{Configuration d’une partie}


\subsubsection{Créer une partie}
Si l’utilisateur sélectionne
"new game", une
fenêtre de configuration de jeu apparaît.
Il peut choisir: \\
    -la taille du plateau\\
    -le nombre de pions\\
    -l'apparition aléatoire de murs sur le plateau\\
    -le ou les autre(s) joueur(s):\\
        \thinspace une ou des IA(s) et sa/leur(s) difficulté(s)\\
        \hskip et/ou un ou des humain(s) qui ont rejoint.
\\A tout moment, il peut démarrer la partie avec les paramètres
affichés
\subsubsection{Rejoindre une partie}

Si l’utilisateur sélectionne
"join", une liste des parties en préparation apparaît.
\\Il peut en sélectionner une pour la rejoindre.
\subsubsection{Gérer une partie}

\subsubsection{Actualisation du classement}

\subsubsection{Logging}


% -----------------------------------------------------
\subsection{Besoins non-fonctionnels}
Contraintes liées au matériel

\subsubsection{Portabilité}

Le programme doit fonctionner sur Linux. Si possible sans trop de modifications, il devrait également être compatible avec Windows.

\subsubsection{Réseau}

Les clients et le serveur doivent être connectés à un même réseau. Lorsque le client se déconnecte subitement du serveur

\subsubsection{Sécurité}

Pour que l'envoi de données sensibles à travers internet ne puisse pas être intercepté, la communication devra se faire à l'aide d'un protocole de communication crypté. Un choix évident serait le TLS.

Une session pourra rester active à l'aide d'un access token.
Cet token aura une date d'expiration. Le programme client gardera ce token dans la RAM.
Ainsi, après le redémarrage du programme client, l'utilisateur devra se reconnecter avec son nom d'utilisateur et son mot de passe.


En ce qui concerne le programme serveur, celui-ci gardera un salted hash du mot de passe dans la base de données.

Récupération de mots de passes: questions secrètes, OTP?

\subsubsection{Fiabilité}

Load balancer, horizontal scaling, ... Then deployment diagram?

Lorsque le client se déconnecte du serveur au cours d'une partie, l'adversaire doit en être averti.
Après une minute d'attente, le joueur déconnecté est déclaré perdant suite à un abandon.


\subsection{Possibilité d'évolution}


\subsection{Légalité}

Do we actually need GDPR complience?

% -----------------------------------------------------
\subsection{Exigences du domaine}

% -----------------------------------------------------
% -----------------------------------------------------
% -----------------------------------------------------

\section{Design et fonctionnement du sytème}

% -----------------------------------------------------
% -----------------------------------------------------
% -----------------------------------------------------

\section{Annexes}
\subsection{Règles de base de Quoridor}
Ces dernières peuvent être consultées sur \href{https://www.gigamic.com/files/catalog/products/rules/quoridor-classic-fr.pdf}{https://www.gigamic.com/files/catalog/products/rules/quoridor-classic-fr.pdf} de son éditeur Gigamic


\end{document}
